\chapter{Exercise 1}

All tasks are implemented in Python using the SciPy-stack.
Image data is parsed into Numpy arrays and processed using a combination of built-in functions from SciPy and functions we have implemented ourselves.

\section*{Task 3 - Basic Image Manipulation}

\subsection*{1 - Correct an image using a flatfield image}

Listing \ref{flatfield} shows the two relevant functions implemented for this task.
For the complete implementation including reading the image files and saving/showing the end result, see the file \texttt{flatfield.py}.

\begin{lstlisting}[language=Python, label=flatfield, caption=Flatfield image correction]
def normalize_bw(img_matrix):
    return map(lambda row: map(lambda val: val / 255, row), img_matrix)


def correct_with_flatfield(img, flatfield):
    corrected = deepcopy(img)
    for y, (a_row, b_row) in enumerate(zip(img, flatfield)):
        for x, (a_px, b_px) in enumerate(zip(a_row, b_row)):
            corrected[y][x] = a_px / b_px

    return corrected
\end{lstlisting}

\subsection*{2 - Speculation}

When performing flat-field correction we want to compensate for imperfections of the detector by scaling the intensity value of a pixel by scaling with the intensity value of the same pixel in the flat-field image.
If we did this by subtraction, i.e original_pixel_intensity - flat_field_intensity, we could end up with negative values.




\section*{Task 4 - Point processing}

\subsection*{1 - Intensity Transform}

ADD STUFF HERE

\section*{Task 5 - Noise}

\subsection*{1 - Salt \& Pepper noise}
