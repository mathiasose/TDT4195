\chapter{Exercise 2}

As with the previous exercise, these tasks were solved using Python/SciPy/Numpy/Matplotlib.
Pillow is still used to load images.

\section*{Task 1 - Filtering}

\subsection*{1.1 - Discrete Convolution}

\begin{lstlisting}[language=Python, label=discrete_convolution, caption=Discrete convolution - box filter]
def list_get(li, i, pad_with=0):
    return li[i] if 0 <= i < len(li) else pad_with


def discrete_convolution(f, g, D):
    return lambda n: sum((f(k) * g(n - k)) for k in D)


def task1_1():
    print('1.1')
    F = [1, 1, 1, 1, 1]
    G = [0, 0, 0, 0, 0, 0, 1, 1, 1, 1, 1, 0, 0, 0, 0, 0, 0]
    f = lambda n: list_get(F, n)
    g = lambda n: list_get(G, n)

    r = len(F) // 2
    D = range(-r, r)

    print(map(discrete_convolution(f, g, D), xrange(len(G))))

>>> task1_1()
1.1
[0, 0, 0, 0, 0, 0, 1, 2, 2, 2, 2, 1, 0, 0, 0, 0, 0]
\end{lstlisting}

Padding? Yes, please.

\subsection*{1.2 - Continuous Convolution}

\begin{lstlisting}[language=Python, label=continous_convolution, caption=Continous convolution]
from scipy.integrate.quadpack import quad
from numpy.core.numeric import Inf
from numpy import linspace

def continuous_convolution(f, g):
    return lambda t: quad(lambda v: (f(v) * g(t - v)), -Inf, Inf)[0]


def task1_2():
    print('1.2')
    f = g = lambda t: 0 if abs(t) > .5 else 1

    values = linspace(-1, 1, 9)
    convoluted = map(continuous_convolution(f, g), values)
    rounded = map(lambda x: round(x, 1), convoluted)
    print(zip(values, rounded))

>>> task1_2()
1.2
[(-1.0, 0.0), (-0.75, 0.2), (-0.5, 0.5), (-0.25, 0.7), (0.0, 1.0), (0.25, 0.7), (0.5, 0.5), (0.75, 0.2), (1.0, 0.0)]
\end{lstlisting}

Which shape do you expect as a result?

\section*{Task 2 - Spatial Filtering}

\subsection*{Adding Noise}
